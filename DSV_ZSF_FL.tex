%%%%%Präambel%%%%%
%Document Style/Page Style
\documentclass[12pt,a4paper]{article}%Schriftgröße, Papierformat einstellen
%\documentclass{scrbook}
\usepackage[top=20mm,bottom=20mm,left=20mm,right=20mm]{geometry}
\usepackage{lipsum}
\usepackage[T1]{fontenc}
\usepackage[ngerman]{babel}
\usepackage[utf8]{inputenc} %für Windows, Linux
\usepackage{titlesec} %improve toc with breaks et cetera
\usepackage{filecontents} %allows to override existing files
\usepackage[nottoc,numbib]{tocbibind}%Add or not add bibliography/index to table of content

%header and footer options
\usepackage{scrpage2} %footers and headers
\pagestyle{scrheadings} %style of headers
\clearscrheadfoot %reset everything
\automark[chapter]{section}
\ofoot{\pagemark}
\ifoot{}
\chead{\headmark}
\setfootsepline{1pt}
\setheadsepline{1pt}
%\setheadsepline[\textwidth+20pt]{0.5pt}

%Citing/Quotation
\usepackage{cite} %improves citation
\usepackage{csquotes}%support for multiple quotation styles
%Pakete laden zur deutschen Rechtschreibung und für Umlaute
\usepackage{harvard} %harvard quotation style
\let\harvardleftorig\harvardleft %option for harvard package

%Symbols
\usepackage[official]{eurosym} %euro symbol
\usepackage{cancel} %diagonal lines for cancelling 
\usepackage{stmaryrd} %St Mary Road symbols for theoretical computer science
\usepackage{units} %set units in typographically correct way
\usepackage{siunitx} %use si units
\usepackage{esvect} %Vectors with arrows
\usepackage{trfsigns} %Transformationszeichen
\usepackage{calc} %adds simple arithmetic functions
\usepackage{amsmath,amssymb,amsthm,amsopn} %mathematic symbols and stuff
\usepackage{bm} %bold symbols in math

%Tables/Environments
\usepackage{longtable} %tables that go more than one page
\usepackage{multirow} %table cell over multiple rows
\usepackage{tabularx} %tabularx environment
\usepackage{booktabs} %provides extra commands for tables
\usepackage{caption} %customise caption in floating environments
\usepackage{array} %use array package
\usepackage{autobreak} %adds line breaking withig align environment
\usepackage{wrapfig} %allows figures or tables to have text wrapped around them
\usepackage{float} %adds options for configuraiton of floationg environments ([H] for example)
\usepackage{hhline} %better horicontal lines

%Misc
\usepackage{graphicx} %optional arguments for includegraphics
\usepackage[dvipsnames]{xcolor} %access to color tones/shades and so on
%\usepackage{subcaption} %test about better image alignment%outdated
\usepackage{subfig}%test about better image alignment

\usepackage{chngcntr} %not quite sure, propably used to specify counting for formulas and stuff
%\usepackage[round]{natbib}
%\usepackage{hyperref}

%Inhaltsverzeichnis mit Links erstellen
\usepackage[colorlinks,
pdfpagelabels,
pdfstartview = FitH,
bookmarksopen = true,
bookmarksnumbered = true,
linkcolor = black,
plainpages = false,
hypertexnames = false,
citecolor = black] {hyperref}

%set depth of toc to specific number
\setcounter{secnumdepth}{4}
\setcounter{tocdepth}{4}

%change how paragraph worls
\titleformat{\paragraph}
{\normalfont\normalsize\bfseries}{\theparagraph}{1em}{}
\titlespacing*{\paragraph}
{0pt}{3.25ex plus 1ex minus .2ex}{1.5ex plus .2ex}

%define subsubsubsection
\newcommand{\subsubsubsection}{\paragraph}

%mainly helping my laziness to flourish

%Makros
%Makro Color
%#1 Text
\def\colBord#1{\begingroup\color{Fuchsia}{#1}\endgroup}
\def\colRed#1{\begingroup\color{Red}{#1}\endgroup}
\def\colGreen#1{\begingroup\color{LimeGreen}{#1}\endgroup}
\def\colBlue#1{\begingroup\color{NavyBlue}{#1}\endgroup}

\def\usGreen#1#2{\underset{\colGreen{#1}}{#2}}
\def\usBord#1#2{\underset{\colBord{#1}}{#2}}

\def\ubGreen#1#2{\underbrace{#2}_{\colGreen{#1}}}

\def\|{\;|\;}
\def\ssum{ \sum_{k=1}^n}

%Some stuff i just dont wanna write everytime
\def\fermi{Fermi-Dirac-Verteilung}
\def\ul#1{\underline{#1}}
\def\€{\euro{}}
\def\epsF{\pmb{\varepsilon}}

\newcommand{\leftRightsection}[2]{
\begin{minipage}[t]{0.5\linewidth}
	\flushleft
	\textbf{#1}
\end{minipage}
	\hfill
\begin{minipage}[t]{0.4\linewidth}
	\flushright
	\textbf{#2}
\end{minipage}
}
\newcommand{\dottedSection}[2]{
\begin{minipage}[t]{0.95\linewidth}
	\textbf{#1 \dotfill #2}
	\end{minipage}
}
%This file contains loosely and not sorted math definitions...

\DeclareMathOperator{\grad}{grad}
\DeclareMathOperator{\diverg}{div}
\DeclareMathOperator{\rot}{rot}
\DeclareMathOperator{\spur}{spur}
\DeclareMathOperator{\determ}{det}

% Umgebungen für Definitionen, Sätze, usw.
\newtheorem{satz}{Satz}[section]
\newtheorem{definition}[satz]{Definition}     
\newtheorem{lemma}[satz]{Lemma}	
\newtheorem{bem}{Bemerkung}[section]
\newtheorem{bsp}{Beispiel}[section]
% Es werden Sätze, Definitionen etc innerhalb einer Section mit
% 1.1, 1.2 etc durchnummeriert, ebenso die Gleichungen mit (1.1), (1.2) ..                  
\numberwithin{equation}{section}


%neue Befehle definieren
\newcommand{\R}{\mathbb{R}} %zB \R als Abkürzung für das Symbol der reellen Zahlen
\newcommand{\N}{\mathbb{N}}
\newcommand{\Z}{\mathbb{Z}}
\newcommand{\Q}{\mathbb{Q}}
\newcommand{\C}{\mathbb{C}}
\newcommand{\diffp}{\partial}
\newcommand{\lapl}{\Delta}
%\def\lapl{\Delta}
%\newcommand{\diverg}{\operatorname{div}}

\def\vecT#1{\left(\begin{array}{c} #1 \end{array}\right)}
\def\dddot{\cdot \\ \cdot \\ \cdot}
\def\vecD#1{\vecT{#1_1 \\ \dddot \\ #1_d}}
\def\vecDt#1#2{\vecT{#1 \\ \dddot \\ #2}}
\def\vecN{\mathcal{O}}
\def\vspan#1{span \lbrace #1 \rbrace}
\def\vdim#1{dim \lbrace #1 \rbrace}
\def\vker#1{ker \lbrace #1 \rbrace}
\def\vrang#1{Rang \lbrace #1 \rbrace}
\def\mzxz#1#2#3#4{\left(\begin{array}{c c} #1 & #2 \\ #3 & #4 \\ \end{array}\right)}
\def\mdxd#1#2#3{\left(\begin{array}{c c c} #1 \\ #2 \\ #3 \end{array}\right)}
\def\dfp#1#2{\frac{\partial #1}{\partial #2}}
\def\diff#1#2{\frac{\mathrm{d}#1}{\mathrm{d}#2}}

\def\fracd#1#2{\displaystyle\frac{#1}{#2}}


\def\inR#1{\qquad ,\; #1 \in \R}
\def\inRs{\in \R}
\def\bracks#1{\left[ #1 \right]}
\def\abs#1{\left| #1 \right|}
\def\brac#1{\left( #1 \right)}
\def\dx{\;dx}
\def\dy{\;dy}
\def\dz{\;dz}
\def\dX{\;d\vec{x}}
\newcommand\citevgl
{\def\harvardleft{(vgl.\ \global\let\harvardleft\harvardleftorig}%
 \cite
}
\newcommand\citeVgl
{\def\harvardleft{(Vgl.\ \global\let\harvardleft\harvardleftorig}%
 \cite
}


\def\ccite#1#2{\glqq #1\grqq\cite{#2}}


\def\ezQu#1{'#1'}

\newcolumntype{L}[1]{>{\raggedleft\let\newline\\\arraybackslash\hspace{0pt}}m{#1}}

\def\multiTwo#1#2{\multicolumn{2}{>{\hsize=\dimexpr2\hsize+2\tabcolsep+\arrayrulewidth\relax}#1}{#2}}
\def\multiThree#1#2{\multicolumn{3}{>{\hsize=\dimexpr3\hsize+4\tabcolsep+2\arrayrulewidth\relax}#1}{#2}}

\newcolumntype{L}[1]{>{\raggedleft\let\newline\\\arraybackslash\hspace{0pt}}m{#1}}
\newcolumntype{R}[1]{>{\raggedright\let\newline\\\arraybackslash\hspace{0pt}}m{#1}}
\newcolumntype{P}[1]{>{\centering\arraybackslash}p{#1}}

\def\formTab#1#2{
\begin{equation}
  \begin{tabularx}{12cm}{R{3cm} l l}
    #1 &: &$#2$
  \end{tabularx}
\end{equation}
}
\newcommand{\formTabL}[3]{
\begin{equation}
  \begin{tabularx}{12cm}{R{3cm} l l}
    #1 &: &$#2$ 
  \end{tabularx}
  \label{eq:#3}
\end{equation}}
\def\formTn{$ \\ $\;$ & $\;$ & $}
\def\formTnQ{$ \\ $\;$ & $\;$ & $\qquad}
\def\formTnQQ{$ \\ $\;$ & $\;$ & $\qquad \qquad}
\def\formTnQQQ{$ \\ $\;$ & $\;$ & $\qquad \qquad \qquad}



%really not sure anymore what this is
\newcommand{\tabitem}{~~\llap{\textbullet}~~}

%settings about equation numbering
\renewcommand{\theequation}{\arabic{section}.\arabic{subsection}
.\arabic{equation}}
%Setzt den equation-Zaehler nach jeder Seite zurueck
%\numberwithin{equation}{subsection}	
\numberwithin{equation}{section}
%\setlength\abovedisplayskip{0pt}

%documentspecific definitions
\newcommand{\ld}{\text{ld}}
\newcommand{\overbar}[1]{\mkern 1.5mu\overline{\mkern-1.5mu#1\mkern-1.5mu}\mkern 1.5mu}

%jetzt beginnt das eigentliche Dokument
\begin{document}
\bibliographystyle{agsm}

\author{}


\null  % Empty line
\nointerlineskip  % No skip for prev line
\vfill
\let\snewpage \newpage
\let\newpage \relax
\title{\underline{Digitale Signalverarbeitung Zusammenfassung} \\ $\;$ \\ $\;$ \\ Florian Leuze}
\date{}
\maketitle
\let \newpage \snewpage
\vfill 
\break % page break


\vspace*{\fill} 
\begin{center}
    \begin{LARGE}
		  \glqq Information is
    \end{LARGE}\\
    \begin{LARGE}
		   the resolution of uncertainty.\grqq
    \end{LARGE}\\
    \begin{large}
      (C. E. Shannon)
    \end{large}
\end{center}
\vspace*{\fill}

\newpage
\tableofcontents

\section*{Versionierung}
\begin{tabular}{|p{2cm}|p{1cm}|p{1.5cm}|p{10.5cm}|}\hline
Datum & Vers. & Kürzel & Änderung \\ \hline
15.10.2019 & 0.1 & FL & Erzeugung Dokument; Erzeugung Inhaltsverzeichnis; Erzeugung Versionierung; \\ \hline
\end{tabular}
\listoffigures


%
\chapter{Einleitung}
\section{Was ist Signalverarbeitung}
	Signalverarbeitung ist die Extraktion eines oder mehrerer Ausgangssignale oder Informationen aus einem oder mehreren Eingangssignalen. Das kann zum Beispiel Rauschunterdrückung sein (Signal $\rightarrow$ Signal) oder die Extraktion der Position eines Sprechers (Signal $\rightarrow$ Information).
	
	\subsection{Methoden der Signalverarbeitung}
	Methoden dieser Vorlesung:
	\begin{itemize}
		\item Filterung
		\item Korrelation
		\item Spektralanalyse
	\end{itemize}
	Methoden weiterer Vorlesungen:
	\begin{itemize}
		\item Parameterschätzung
		\item Optimalfilter und adaptive Filter
		\item Detektion und Mustererkernnung 
		\item $\dots$
	\end{itemize}
	
	\subsection{Digitale Signalverarbeitung}
	\begin{center}
		\begin{tikzpicture}	
			% Place nodes using a matrix
			\matrix[row sep=2.5mm, column sep=20mm]
			{
				%--------------------------------------------------------------------
				\node[dspnodeopen,dsp/label=above] (m00) {$x_a(t)$};    &
				\node[dspsquare]                   (m01) {$\;$ ADW $\;$}; &
				\node[dspsquare]                   (m02) {$\;$ $\;$DSV$\;$ $\;$}; &
				\node[dspsquare]                   (m03) {$\;$ DAW $\;$}; &
				\node[dspnodeopen,dsp/label=above] (m04) {$y_a(t)$};    \\
				%--------------------------------------------------------------------
			};
		
			% Draw connections
			
			\begin{scope}[start chain]
				\chainin (m00);
				\path (m00) edge node [below] {\parbox{1.8cm}{Analoger \\ Eingang}} (m01) ;
				\chainin (m01) [join=by dspconn];
				\path (m01) edge node [above] {$x_q (n)$} (m02) ;
				\path (m01) edge node [below] {\parbox{1.8cm}{Digitaler \\ Eingang}} (m02) ;
				\chainin (m02) [join=by dspconn];
				\path (m02) edge node [above] {$y_q (n)$} (m03) ;
				\path (m02) edge node [below] {\parbox{1.8cm}{Digitaler \\ Ausgang}} (m03) ;
				\chainin (m03) [join=by dspconn];
				\path (m03) edge node [below] {\parbox{1.8cm}{Analoger \\ Ausgang}} (m04) ;
				\chainin (m04) [join=by dspconn];
			\end{scope}
		\end{tikzpicture}
	\end{center}
	
	\subsection{Vor- und Nachteile}
	\begin{description}
		\item[+] Beliebig hohe Genauigkeit
		\item[+] Einfache Speicherung und Transport von Daten
		\item[+] Flexibel (ein Prozessor für viele Aufgaben)
		\item[+] Beliebig komplexe Signalverarbeitung möglich
		\item[+] Billig für Massenprodukte
		\item[-] Ungeeignet für extrem hohe Bandbreiten
	\end{description}
	
	\subsection{Unschärferelation der digitalen Signalverarbeitung}
	\begin{equation}
		\text{Abtastfrequenz } \times\text{ Komplexität der Signalverarbeitung pro Abtastwert} \leq \text{verfügbare Rechenleistung}
	\end{equation}
	Oder in Worten ausgedrückt, wie viele Abtastwerte können von der Hardware verarbeitet werden.
\section{Analog-Digital-Wandler (ADW)}
	\begin{center}
		\begin{tikzpicture}[scale=0.9, every node/.style={scale=0.9}]
    			\draw[dspconn] (0,0) -- node[above]{$x_a(t)$} (1.5,0);
    			\draw[] (0,0) -- node[below]{\parbox{4cm}{\centering \vspace{+0.5cm} Analog: \\ zeitkontinuierlich \\ wertkontinuerlich}} (1.5,0);
    			\draw (1.5,0.5) rectangle (3.7,-0.5) node[pos=0.5]{Abtastung};
    			\draw[dspconn] (3.7,0) -- node[above]{$x(n)$} (5.2,0);
    			\draw[] (3.7,0) -- node[below]{\parbox{4cm}{\centering  \vspace{+1cm}  zeitdiskret \\ wertkontinuerlich}} (5.2,0);
    			\draw (5.2,0.5) rectangle (8,-0.5) node[pos=0.5]{Quantisierung};
    			\draw[dspconn] (8,0) -- node[above]{$x_q(n)$} (9.5,0);
    			\draw[] (8,0) -- node[below]{\parbox{4cm}{\centering \vspace{+0.5cm} Digital: \\ zeitdiskret \\ wertdiskret}} (9.5,0);
 	  \end{tikzpicture}
	\end{center}
	
	\subsection{Wichtige Kenndaten eines ADW}
	\begin{description}
		\item[$\ast$) ]Abtastfrequenz: Hz $\sim$ GHz
		\item[$\ast$) ]Wortbreite: $8 \sim 16$ Bits
	\end{description}
	
	\subsection{Quantisierung}
		Approximation von $x(n) \in \R$ durch $b$ Bits:
		\begin{equation}
			x_q(n) = Q[x(n)] = x(n) + e(n)
		\end{equation}
		mit: \\ $\;$\\
		\begin{tabular}{l c l}
		$b$& : & Wortbreite, $2^b$ mögliche Werte für $x_q(n)$ \\
		$Q$& : & Quantisierungsoperator\\
		$e(n)$& : & Quantisierungsfehler,- rauschen (zufällig) $\sim \frac{1}{2^b}$ \\
		\end{tabular} \\ $\;$ \\
		wobei die Stufenbreite die Amplitude des Quantisierungsfehlers bestimmt.
		\subsection{Signal-Rausch-Abstand (SNR)}
		\begin{align}
			SNR[dB] &= 10 \cdot \log_{10} \frac{\text{Signalleistung}}{\text{Rauschleistung}} \nonumber\\
				&= 10 \cdot \log_{10} \frac{\overbar{x^2(t)}}{\overbar{e^2(t)}} = \dots = \text{const} + 10 \cdot log_{10}\left( 2^b \right)^2 \nonumber \\
				&\approx \text{const} + 6 \cdot b
		\end{align}
		wobei $x(t)$ die Leistung des Nutzsignals ist. Damit folgt direkt ein Anstieg des $SNR$ um $6dB$ pro zusätzlichem Bit.
		
		\subsection{Digital-Analog-Wandler (DAC)}
\newpage
\section{Anhänge}
	\subsection{Nachwort}
Dieses Dokument versteht sich einzig als Zusammenfassung des NT2 Stoffes auf Basis der Literatur und der Vorlesungsunterlagen aus der NT2 Vorlesung von Prof. Dr.-Ing Stephan ten Brink. Der Sinn ist einzig mir selbst und meinen Kommilitonen das studieren der Nachrichtentechnik zu erleichtern. In diesem Sinne erhebe ich keinerlei Anspruch auf das hier dargestellte Wissen, da es sich in großen Teilen nur um Neuformulierungen aus der Literatur, den Vorlesungen und aus der Begleitübung. Sollten sich einige Fehler eingeschlichen haben (was sehr wahrscheinlich ist) würde ich mich freuen, wenn man mich per Email (f.leuze@outlook.de) kontaktieren und mir entsprechende Fehler mitteilen würde.

\nocite{*}
\bibliography{./bib/lit}

\end{document}
