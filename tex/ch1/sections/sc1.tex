\section{Was ist Signalverarbeitung}
	Signalverarbeitung ist die Extraktion eines oder mehrerer Ausgangssignale oder Informationen aus einem oder mehreren Eingangssignalen. Das kann zum Beispiel Rauschunterdrückung sein (Signal $\rightarrow$ Signal) oder die Extraktion der Position eines Sprechers (Signal $\rightarrow$ Information).
	
	\subsection{Methoden der Signalverarbeitung}
	Methoden dieser Vorlesung:
	\begin{itemize}
		\item Filterung
		\item Korrelation
		\item Spektralanalyse
	\end{itemize}
	Methoden weiterer Vorlesungen:
	\begin{itemize}
		\item Parameterschätzung
		\item Optimalfilter und adaptive Filter
		\item Detektion und Mustererkernnung 
		\item $\dots$
	\end{itemize}
	
	\subsection{Digitale Signalverarbeitung}
	\begin{center}
		\begin{tikzpicture}	
			% Place nodes using a matrix
			\matrix[row sep=2.5mm, column sep=20mm]
			{
				%--------------------------------------------------------------------
				\node[dspnodeopen,dsp/label=above] (m00) {$x_a(t)$};    &
				\node[dspsquare]                   (m01) {$\;$ ADW $\;$}; &
				\node[dspsquare]                   (m02) {$\;$ $\;$DSV$\;$ $\;$}; &
				\node[dspsquare]                   (m03) {$\;$ DAW $\;$}; &
				\node[dspnodeopen,dsp/label=above] (m04) {$y_a(t)$};    \\
				%--------------------------------------------------------------------
			};
		
			% Draw connections
			
			\begin{scope}[start chain]
				\chainin (m00);
				\path (m00) edge node [below] {\parbox{1.8cm}{Analoger \\ Eingang}} (m01) ;
				\chainin (m01) [join=by dspconn];
				\path (m01) edge node [above] {$x_q (n)$} (m02) ;
				\path (m01) edge node [below] {\parbox{1.8cm}{Digitaler \\ Eingang}} (m02) ;
				\chainin (m02) [join=by dspconn];
				\path (m02) edge node [above] {$y_q (n)$} (m03) ;
				\path (m02) edge node [below] {\parbox{1.8cm}{Digitaler \\ Ausgang}} (m03) ;
				\chainin (m03) [join=by dspconn];
				\path (m03) edge node [below] {\parbox{1.8cm}{Analoger \\ Ausgang}} (m04) ;
				\chainin (m04) [join=by dspconn];
			\end{scope}
		\end{tikzpicture}
	\end{center}
	
	\subsection{Vor- und Nachteile}
	\begin{description}
		\item[+] Beliebig hohe Genauigkeit
		\item[+] Einfache Speicherung und Transport von Daten
		\item[+] Flexibel (ein Prozessor für viele Aufgaben)
		\item[+] Beliebig komplexe Signalverarbeitung möglich
		\item[+] Billig für Massenprodukte
		\item[-] Ungeeignet für extrem hohe Bandbreiten
	\end{description}
	
	\subsection{Unschärferelation der digitalen Signalverarbeitung}
	\begin{equation}
		\text{Abtastfrequenz } \times\text{ Komplexität der Signalverarbeitung pro Abtastwert} \leq \text{verfügbare Rechenleistung}
	\end{equation}
	Oder in Worten ausgedrückt, wie viele Abtastwerte können von der Hardware verarbeitet werden.